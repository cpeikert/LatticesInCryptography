\documentclass[11pt]{article}

\usepackage{import}
\import{common/}{head.tex}
\import{common/slides/}{tikzhead.tex}
\definecolor{structure.fg}{rgb}{0.2,0.2,0.7} % from favorite beamer theme

%%% COMMANDS FOR LECTURES/HOMEWORKS

\RequirePackage{fancyhdr}

\newcommand{\lecheader}{%
  \chead{\large \textbf{Lecture \lecturenum\\\lecturetopic}}

  \lhead{\small
    \textbf{\href{https://github.com/cpeikert/LatticesInCryptography}{Lattices In Cryptography}}\\\textbf{Winter 2026}}

  \rhead{\small \textbf{Instructor:
      \href{http://www.eecs.umich.edu/~cpeikert/}{Chris Peikert}\\Scribe:
      \scribename}}

  \setlength{\headheight}{20pt}
  \setlength{\headsep}{16pt}
}

%%% xsim settings

\RequirePackage{xsim}
\RequirePackage{needspace}

\DeclareExerciseEnvironmentTemplate{QRef}{%
    \par\addvspace{\baselineskip}
    \Needspace*{2\baselineskip}
    \noindent
    \hyperref[ans:\GetExerciseProperty{ID}]{\textbf{\XSIMmixedcase{\GetExerciseName}~\GetExerciseProperty{counter}.}}
    \label{ques:\GetExerciseProperty{ID}}
}{\par\addvspace{\baselineskip}}

\DeclareExerciseEnvironmentTemplate{ARef}{%
        \Needspace*{2\baselineskip}
        \noindent
        \hyperref[ques:\GetExerciseProperty{ID}]{\textbf{\XSIMmixedcase{\GetExerciseParameter{exercise-name}}~\GetExerciseProperty{counter}.}}
        \label{ans:\GetExerciseProperty{ID}}
        \GetExerciseBody{exercise}
        \par \medskip \noindent
        \textbf{\XSIMmixedcase{\GetExerciseName}.}
}{\par\addvspace{\baselineskip}}

\DeclareExerciseHeadingTemplate{simple}{%
    \section*{\XSIMmixedcase{\GetExerciseParameter{solution-name}s}}
}

\DeclareExerciseType{question}{%
    exercise-env = question,
    solution-env = answer,
    exercise-name = Question,
    solution-name = Answer,
    exercise-template = QRef,
    solution-template = ARef,
}

\xsimsetup{%
  path = {questions},
  file-extension = {aux},
  print-solutions/headings-template=simple,
  exercise/template=runin
}

\loadxsimstyle{layouts}

\AtEndDocument{\pagebreak\printsolutions}


\chead{\large \textbf{Course Information\\and Syllabus}}

\lhead{\small
  \textbf{\href{https://github.com/cpeikert/LatticesInCryptography}%
    {Lattices in Cryptography}\\University of Michigan, \href{https://umich.instructure.com/courses/823128}{Winter 2026}}}

\rhead{\small \textbf{Instructor:
    \href{http://www.eecs.umich.edu/~cpeikert/}{Chris Peikert}}}

\setlength{\headheight}{27pt}
\setlength{\headsep}{20pt}

\pagestyle{plain}               % default: no special header

\begin{document}

\thispagestyle{fancy}           % first page should have special header

\section{General Information}%
\label{sec:general-information}

\emph{Point lattices} in $\R^{n}$ have proven remarkably useful in cryptography, both for \emph{cryptanalysis} (breaking codes) and more recently for \emph{constructing} cryptosystems with desirable security, efficiency, and functionality properties.

This graduate-level seminar will cover classical results, exciting recent developments, and important open problems.
Specific topics, depending on time and level of interest, will be drawn from:
\begin{itemize}[itemsep=0pt]
\item Mathematical background and basic results
\item The LLL algorithm, Coppersmith's method, and applications to cryptanalysis
\item Complexity of lattice problems: NP-hardness, algorithms and other upper bounds
\item Gaussians, harmonic analysis, and the smoothing parameter
\item Worst-case-to-average-case reductions and the SIS/LWE problems
\item Basic cryptographic constructions: one-way functions, encryption schemes, digital signatures
\item ``Advanced'' cryptographic constructions: ID-based encryption, fully homomorphic encryption, etc.
\item ``Algebraic'' (ring-based) hardness reductions and cryptographic primitives
\end{itemize}

\subsection{Materials}%
\label{sec:materials}

The public course web page with lecture notes and other materials is at {\small \url{https://github.com/cpeikert/LatticesInCryptography}}.
For assignments and submission, grading, discussions, etc., we will use the course Canvas site at \url{https://umich.instructure.com/courses/823128}.
For brief questions and answers, we will use the course Piazza site at \url{https://piazza.com/umich/winter2026/eecs598002}.

There is no required textbook for this class; lectures, notes, and research papers are the main sources of instructional material.
Students may also wish to refer to the following excellent sources:
\begin{itemize}[itemsep=0pt]
\item Oded Regev's course \href{http://www.cims.nyu.edu/~regev/teaching/lattices_fall_2009/index.html}{\emph{Lattices in Computer Science}}
\item Micciancio and Goldwasser's book \href{http://link.springer.com/book/10.1007/978-1-4615-0897-7/page/1}{\emph{Complexity of Lattice Problems: A Cryptographic Perspective}}
\end{itemize}

Instructor office hours will be held in Beyster 3601 on \textbf{Fridays at 9am} (with a couple exceptions, to be announced ahead of time), or by appointment.

\subsection{Prerequisites}%
\label{sec:prerequisites}

There are no formal prerequisite classes.
However, this course is mathematically rigorous and fast-paced, hence the main requirement is \emph{mathematical maturity}.
Specifically, students should be comfortable with devising and writing correct formal proofs (and finding the flaws in incorrect ones!), devising and analyzing algorithms, and working with probability.

A previous course in cryptography (e.g., Intro/Advanced Cryptography, EECS 475/575) is very helpful but is not required.
No previous familiarity with lattices will be assumed.
Other courses---the more the better---include: EECS~477 or~586 (Algorithms), EECS~574 (Computational Complexity Theory), or any proof-based Mathematics courses in Linear Algebra, Abstract Algebra, etc.
The instructor reserves the right to limit enrollment to students who have the necessary background.

\subsection{Open Problems Session}%
\label{sec:open-problems-session}

In addition to lectures, there is an optional ``open problems session'' held \textbf{Thursdays 11:30am-1:30pm} in CSRB 2236.
During these sessions we will work in small groups on a range of open research problems of interest.
(Your instructor has several interesting problems in mind already, and more will surely emerge during class.)
Some of our efforts may result in quality publications, but regardless of outcomes, it will be a good exposure to research in cryptography and theory.

You may attend as many or as few of these sessions as you like.
You may also register for the associated 1-credit ``lab section'' EECS 598-021, which will be graded based on attendance and participation.

\section{Course Policies}%
\label{sec:policies}

\subsection{Grading}%
\label{sec:grading}

Grades will be determined roughly as follows:
\begin{itemize}
\item[(50\%)] Homework assignments (about 4), due approximately every two weeks.
  Collaboration and external sources are allowed and encouraged; see academic honesty policy for details.

\item[(30\%)] Research-oriented project and presentation.

\item[(20\%)] Participation (including scribe notes) and homework peer review.
\end{itemize}

All submitted work will be graded on \emph{correctness} and \emph{clarity}, and must be typeset in \LaTeX\ (templates will be made available).
It is good practice to start any longer solution with an informal (but accurate) ``proof summary'' that describes the core idea, which will help the reader (and you!)
understand your solution better.

There are no predetermined score thresholds for A/B/C/etc
Your primary focus should be on \emph{learning the material}, not your grade.

\subsection{Academic Honesty}%
\label{sec:academic-honesty}

On homework assignments, collaboration and consultation with external sources is allowed and encouraged, subject to the following conditions:
\begin{itemize}[itemsep=0pt]
\item You must submit your own individually written solution, and you must list your collaborators and/or external sources for each problem.
\item You may not submit a problem solution that you cannot explain orally.
\end{itemize}

There is no hard-and-fast list of (dis)honest conduct.
When in doubt, err on the side of caution, or ask the instructor.
Dealing with academic dishonesty is unpleasant for everyone involved, so please follow these policies!

\end{document}

%%% Local Variables: 
%%% mode: latex
%%% TeX-master: t
%%% End: 
