\documentclass[11pt]{article}

\usepackage{fullpage}
\usepackage{newtxtext}
\usepackage{microtype}
\usepackage{amsmath,amsfonts,amssymb}
\usepackage[amsmath,amsthm,thmmarks,hyperref]{ntheorem}
\usepackage[pagebackref=true]{hyperref} % must set backref at load time
\usepackage{fancyhdr}
\usepackage{mathtools}
\usepackage{enumitem}
\usepackage[svgnames]{xcolor}
\usepackage{tikz}
\usepackage{import}

% load after hyperref, algpseudocode to get proper behavior
\usepackage[capitalize,nameinlink,noabbrev]{cleveref}

\import{common/}{head.tex}

% VARIABLES

\newcommand{\lecturenum}{9}
\newcommand{\lecturetopic}{Fourier Analysis, Gaussians}
\newcommand{\scribename}{Tung Mai}

% END OF VARIABLES

\import{common/}{lechead.tex}
\lecheader                      % execute lecture commands

\pagestyle{plain}               % default: no special header

\begin{document}

\thispagestyle{fancy} % first page should have special header

% LECTURE MATERIAL STARTS HERE

\section{Fourier transform and Fourier series in $n$-dimension}

Recall from last lecture that in the one dimensional case, the Fourier
transform of a function $f \colon \R \to \C$ is the function
$\hat{f} \colon \R \to \C$ defined as
\[ \hat{f}(w) = \int_{x\in \R} f(x) \exp(-2 \pi ixw) \, dx . \] The
inversion formula is
\[ f(x) = \int_{w\in \R} \hat{f}(w) \exp(2 \pi ixw) \, dx. \] For a
$\Z$-periodic function $g \colon (\R / \Z) \to \C$, its Fourier series
is the function $\hat{g} \colon \Z \to \C$ defined as
\[ \hat{g}(w) = \int_{x \in \R/\Z} g(x) \exp(-2 \pi ixw) \, dx . \] The
inversion formula is
\[ g(x+\Z) = \sum_{w \in \Z} \hat{g}(w) \exp(2 \pi ixw). \]

We now extend the Fourier transform and Fourier series to~$n$
dimensions. Similarly to before, define $L^{1}(\R^n)$ to be the set of
functions $f \colon \R^n \to \C$ for which
$\int_{\R^n} \abs{f(\vecx)} \, d\vecx < \infty$.

\begin{definition}
  \label{def:fourier-transform}
  For $f \in L^{1}(\R^n)$, the Fourier transform of~$f$ is the
  function $\hat{f} \colon \R^{n} \to \C$ defined as
  \[ \hat{f}(\vecw) = \int_{\vecx\in \R^n} f(\vecx)\exp(-2 \pi i
    \inner{\vecx,\vecw}) \, d\vecx . \]
\end{definition}

\begin{definition}
  For a $\Z^{n}$-periodic function $g \colon \R^{n}/\Z^{n} \to \C$,
  its Fourier series $\hat{g} \colon \Z^{n} \to \C$ is defined as
  \[ \hat{f}(\vecw) = \int_{\vecx\in \R^n / \Z^n} f(\vecx) \exp(-2 \pi
    i\inner{\vecx,\vecw}) \, d\vecx . \]
\end{definition}

We now mention some easy properties of the $n$-dimensional Fourier
transform and Fourier series (where applicable); their proofs
naturally generalize from the one-dimensional case.
\begin{enumerate}
\item \emph{Linearity:} $\widehat{f+g} = \hat{f}+\hat{g}$ and
  $\widehat{c \cdot f} = c \cdot \hat{f}$ for any $c \in \R$.
\item \emph{Shift property:} if $h(\vecx) = f(\vecx - \vecc)$ for some
  $\vecc \in \R^{n}$ then
  $\hat{h}(\vecw) = \exp(-2 \pi i \inner{\vecc,\vecw}) \cdot
  \hat{f}(\vecw)$.
\item \emph{Linear transform property:} if
  $h(\vecx) = f(\matB \vecx)$ for some nonsingular
  $\matB \in \R^{n \times n}$, then
  $\hat{h}(\vecw) = \frac{1}{\det(\matB)} \hat{f}(\matB^{-t}\vecw)$.
  Here $\matB^{-t} = (\matB^{-1})^t = (\matB^{t})^{-1}$.
  \begin{proof}
    From the definition of Fourier transform, we have
    \[ \hat{h}(\vecw) = \int_{\vecx\in \R^n} f(\matB\vecx)\exp(-2 \pi
      i \inner{\vecx,\vecw}) \, d\vecx.\] Letting
    $\vecu = \matB \vecx$, we have $d\vecu = \det(\matB) \, d\vecx$,
    and
    $\inner{\vecx,\vecw} = \vecx^t \cdot \vecw = (\vecx^t \matB^t)
    \cdot (\matB^{-t}\vecw) = \inner{\matB \vecx, \matB^{-t}\vecw}$. So
    \begin{align*}
      \hat{h}(\vecw) &= \int_{\vecu \in \R^n} f(\vecu)\exp(-2 \pi i \inner{\vecu, \matB^{-t}\vecw} ) \frac{1}{\det(\matB)} \, d\vecu  \\
                     &= \frac{1}{\det(\matB)} \hat{f}(\matB^{-t}\vecw).
    \end{align*}
  \end{proof}

\item \emph{Poisson summation formula:} $f(\Z^{n}) = \hat{f}(\Z^n)$.
\end{enumerate}

\section{Dual Lattices and $\lat$-Periodic Functions}

\subsection{Definitions}

So far, the periodic functions we have considered have only been
$\Z^n$-periodic. We now extend the notion of Fourier series to
functions that are periodic over a lattice~$\lat$, namely functions
$g \colon \R^n/ \lat \to \C$. One approach is to transform~$g$ to a
$\Z^n$-periodic function~$h$. Letting~$\matB$ be a basis of~$\lat$, we
can write $\lat = \matB \Z^n$. Since~$g$ is $\lat$-periodic,
$h(\vecx) := g(\matB \vecx)$ is $\Z^n$-periodic. We can find the
Fourier series for~$h$ and then use the scaling property to obtain the
Fourier series for~$g$. However, this approach requires switching back
and forth between a $\lat$-periodic function and a $\Z^n$-periodic
function, which can be cumbersome. Therefore, we show another
approach, which is to define the Fourier series for~$g$ directly. For
this we need the notion of the \emph{dual lattice}.

\begin{definition}[Dual lattice]
  \label{def:dual-lattice}
  For a lattice $\lat \subset \R^{n}$, its dual lattice $\lat^{*}
  \subset \R^{n}$ is defined as
  \begin{align*}
    \lat^{*}
    &= \set{ \vecw : \inner{\vecv,\vecw} \in \Z \; \forall\;
      \vecv \in \lat} \\
    &= \set{ \vecw : \inner{\lat,\vecw} \subseteq \Z} .
  \end{align*}
\end{definition}

\begin{definition}
  For a lattice $\lat \subset \R^{n}$ and a function
  $g \colon \R^n/\lat \to \C$, its Fourier series $g \colon \lat^{*}
  \to \C$ is defined as
  \[ \hat{g}(\vecw) = \frac{1}{\det(\lat)} \int_{\vecx \in \R^n /
      \lat} g(\vecx) \exp(-2 \pi i \inner{\vecx,\vecw} )\ d\vecx .\]
\end{definition}
Notice that~$\vecx$ is a \emph{coset} $\vecc+\lat$ for some
$\vecc \in \R^{n}$; because
$\inner{\vecx,\vecw} = \inner{\vecc+\lat,\vecw} \subseteq
\inner{\vecc,\vecw} + \Z$, the phase term
$\exp(-2 \pi i \inner{\vecx,\vecw}) = \exp(-2 \pi i \inner{\vecc,
  \vecw})$ is well defined, and invariant under the choice of~$\vecc$
from the coset.

\subsection{Properties of the Dual Lattice}

We show some basic properties of the dual lattice.
\Cref{def:dual-lattice} defines the dual of~$\lat$ as the set of
points whose inner product with any point in~$\lat$ is an integer. The
following claim establishes that~$\lat^*$ actually is a lattice.

\begin{claim}
  \label{clm:dual-basis}
  If~$\matB$ is a basis of~$\lat$, then~$\matB^{-t}$ is a basis
  of~$\lat^*$.
\end{claim}

\begin{proof}
  We show that $\lat^{*} = \lat(\matB^{-t})$ by proving inclusions in
  both directions. For any $\vecw = \matB^{-t} \vecz$ where
  $\vecz \in \Z^n$,
  \[ \inner{\matB \cdot \Z^n, \vecw} = \inner{\matB \cdot
      \Z^n,\matB^{-t} \vecz} = \inner{\Z^n,\vecz} \subseteq \Z. \] So
  $\lat(\matB^{-t}) \subseteq \lat^*$. In the other direction, for any
  $\vecw \in \lat^*$, we have $\vecz := \matB^{t}\vecw \in \Z^{n}$
  (because the columns of~$\matB$ are vectors in~$\lat$), so
  $\vecw = \matB^{-t} \vecz \in \lat(\matB^{-t})$, hence
  $\lat^* \subseteq \lat(\matB^{-t})$.
\end{proof}

\begin{claim}
  For any lattice~$\lat$, we have $(\lat^*)^* = \lat$.
\end{claim}

\begin{proof}
  By \cref{clm:dual-basis}, a basis of $(\lat^*)^*$ is
  $(\matB^{-t})^{-t} = \matB$. Therefore, $(\lat^*)^{*} = \lat$, since
  they are generated by the same basis.
\end{proof}

\begin{claim}
  \label{clm:dual-determinant}
  For any lattice~$\lat$, we have $\det(\lat^*) = 1/\det(\lat)$.
\end{claim}

\begin{proof}
  Since $\matB$ is a basis of $\lat$, and $\matB^{-t}$ is a basis of
  $\lat^*$,
  \[ \det(\lat^*) = \abs{\det(\matB^{-t})} = \frac{1}{\abs{\det(\matB)}} =
    \frac{1}{\det{\lat}}.\]
\end{proof}

\begin{claim}
  For any $n$-dimensional lattice~$\lat$, we have
  $\lambda_1(\lat) \cdot \lambda_1(\lat^*) \leq n$.
\end{claim}

\begin{proof}
  By Minkowski's inequality we have
  $\lambda_1(\lat) \leq \sqrt{n} \det(\lat^{1/n})$ and
  $\lambda_1(\lat^*) \leq \sqrt{n}\det(\lat^*)^{1/n}$, so by
  \cref{clm:dual-determinant},
  \[ \lambda_{1}(\lat) \cdot \lambda_{1}(\lat^{*}) \leq n \cdot
    \det(\lat)^{1/n} \cdot \det(\lat^*)^{1/n} \leq n . \]
\end{proof}

\subsection{Properties of the Fourier Series}

We mention two important properties of the Fourier series of
$\lat$-periodic functions.

\paragraph{Inversion formula.}

For any $\lat$-periodic function $g \colon \R^{n}/\lat \to \C$, we
have
\[ g(\vecx) = \sum_{\vecw \in \lat^*} \hat{g}(\vecw) \exp(2 \pi i
  \inner{\vecx,\vecw}). \]

\paragraph{Periodization.}

Generalizing the one-dimensional case, we can ``periodize'' a function
by a lattice, and then establish a link between the Fourier transform
and Fourier series, respectively. Let $f \in L^{1}(\R^{n})$, and for a
countable set~$S$, define $f(S) := \sum_{x \in S} f(x)$. For a lattice
$\lat \subset \R^{n}$, periodize~$f$ by summing all its
$\lat$-translates, i.e., define $g \colon \R^{n}/\lat \to \C$ as
\begin{equation}
  \label{eq:periodize}
  g(\vecx + \lat) := f(\vecx + \lat) = \sum_{\vecv \in \lat} f(\vecx +
  \vecv) .
\end{equation}

\begin{lemma}
  \label{lem:periodize-series}
  The Fourier series of~$g$ is
  $\hat{g}(\vecw) = \hat{f}(\vecw) / \det(\lat) = \det(\lat^{*})
  \hat{f}(\vecw)$.
\end{lemma}

\begin{proof}
  Let $\cal{F}$ be any fundamental region of~$\lat$. Then for any
  $\vecw \in \lat^{*}$, we have
  \begin{align*}
    \hat{g}(\vecw)
    &= \frac{1}{\det(\lat)} \int_{\vecx \in \R^n/\lat} g(\vecx) \exp(-
      2 \pi i  \inner{\vecx,\vecw}) d\vecx  \\
    &= \frac{1}{\det(\lat)} \int_{\vecc \in \cal{F}} g(\vecc + \lat) \exp(-
      2 \pi i  \inner{\vecc,\vecw}) d\vecx  \\
    &= \frac{1}{\det(\lat)} \int_{\vecc \in \cal{F}} \sum_{\vecv \in
      \lat} f(\vecc + \vecv) \exp(- 2 \pi i \inner{\vecc +
      \vecv,\vecw}) d\vecx \\
    &= \frac{1}{\det(\lat)} \int_{\vecu \in \R^{n}} f(\vecu) \exp(-2
      \pi i \inner{\vecu,\vecw}) d\vecu \\
    &= \frac{1}{\det(\lat)} \hat{f}(\vecw).
  \end{align*}

\end{proof}

\begin{lemma}[Poisson Summation Formula]
  \label{lem:psf}
  For any ``nice enough'' (differentiable, continuous, \ldots)
  function $f \colon \R^n \to \C$ and any lattice
  $\lat \subset \R^{n}$, we have
  \[ f(\lat) = \hat{f}(\lat^{*}) / \det(\lat) = \det(\lat^*) \cdot
    \hat{f}(\lat^*). \]
\end{lemma}

\begin{proof}
  Let~$g$ be the $\lat$-periodization of~$f$. By the inversion formula
  and \cref{lem:periodize-series},
  \[ f(\lat) = g(\mathbf{0}) = \sum_{\vecw \in \lat^*}
    \hat{g}(\vecw)\exp(2 \pi i \inner{\mathbf{0},\vecw}) = \sum_{\vecw
      \in \lat^*} \hat{g}(\vecw) = \sum_{\vecw \in \lat^{*}}
    \hat{f}(\vecw)/\det(\lat) = \hat{f}(\lat^{*}) / \det(\lat). \]
\end{proof}

\section{Application}

In this section, we discuss an application of high dimensional Fourier
transform, which provides a way of distinguishing ``close'' points
from ``far'' points away from a lattice $\lat$, using a ``hint.''

\paragraph{Problem definition.}

Given~$\lat$, we preprocess it to a hint~$W$, which allows us to later
answer queries: ``Given a point $\vecx$, is $\dist(\vecx,\lat) \leq 1$
or $\dist(\vecx,\lat) \geq \sqrt{n}$?''

\paragraph{Strategy.}

Our strategy is to find a $\lat$-periodic function $f$ such that:
\begin{enumerate}
\item $f(x) \geq 1/1000$ for $d(\vecx,\lat)\leq 1$.
\item $f(x) \leq 2^{-n}$ for $d(\vecx,\lat)\geq \sqrt{n}$.
\item $f$ can be succinctly represented, and there exists a good
  approximation to compute $f$.
\end{enumerate}
Our main tool to create a function $f$ is high dimensional Fourier
transform and series. However, we need another tool, namely, the
Gaussian function.

\begin{definition}
  Define the Gaussian function
  $\rho(\vecx) = \exp(-\pi \length{\vecx}^2) = \exp(-\pi
  \inner{\vecx,\vecx})$.
\end{definition}

\begin{definition}
  
\end{definition}
Define
\[ f(\vecx) = \frac{\sum_{\vecv \in
      \lat}{\rho(\vecx-\vecv)}}{\rho(\lat)} =
  \frac{\rho(\vecx+\lat)}{\rho(\lat)}. \] Note that~$f$ is
$\lat$-periodic. The expression $\rho(\vecx+\lat)$ can be thought of
as sum of the weights at $\mathbf{0}$ of the Gaussian functions
centered at points in the coset $\vecx + \lat$. Intuitively, when
$\vecx$ is not close to any lattice point, $\mathbf{0}$ is not close
to any point in the coset $\vecx + \lat$, and the tail weights at
$\mathbf{0}$ are all small. When $\vecx$ is close to a lattice point,
$\mathbf{0}$ is close to a point in the coset $\vecx + \lat$, and the
weight of the Gaussian function centered at that point is large. In
the next part of the lecture, we will show that $f$ actually satisfies
our above three requirements. First, we have the following easy
claims:

\begin{claim}
  $\rho(\vecx + \lat) \leq \rho(\lat)$ for any $\vecx \in \R^n.$
\end{claim}

\begin{proof}
  \begin{align*}
    \rho(\vecx + \lat)
    &= \det(\lat^*) \sum_{\vecw \in \lat^*} \hat{\rho}(\vecw)\exp(2 \pi \inner{\vecx,\vecw} ) \\
    &\leq   \det(\lat^*)\sum_{\vecw \in \lat^*} \rho(\vecw) \\
    &=  \rho(\lat).
  \end{align*}
  Note that we used Poisson Summation Formula and the fact that
  $\hat{\rho}(\vecw) = \rho(\vecw)$ for Gaussian function.
\end{proof}

\begin{claim}
  \label{claim2}
  For any $s \geq 1$, $\rho_s(\lat) \leq s^n \rho(\lat)$, where
  $\rho_s(\vecx) = \rho(\vecx/s)$.
\end{claim}

\begin{proof}
  From Poisson Summation Formula:
  \[\rho_s(\lat) = \det(\lat^*)\hat{\rho}_{s}(\lat^*).\]
  Since $\hat{\rho}_s = s^n \rho_{1_s}$, we can write:
  \[\rho_s(\lat) = \det(\lat^*)s^n\rho_{1/s}(\lat^*).\]
  Notice that $\rho_{1/s} \leq \rho_{1}$ for $s \geq 1$, therefore:
  \[\rho_s(\lat) \leq \det(\lat^*)s^n\rho(\lat^*).\]
  Now apply the Poisson Summation Formula again,
  $\rho(\lat) = \det(\lat^*)\hat{\rho}(\lat^*) =
  \det(\lat^*)\rho(\lat^*)$, we have
  $\rho(\lat^*) = \rho(\lat)/\det(\lat^*)$. Therefore:
  \[\rho_s(\lat) \leq s^n\rho(\lat).\]
\end{proof}
Next, we state three lemmas whose results directly show that
function~$f$ that we defined meets our requirements.

\begin{lemma}
  $f(\vecx) \geq \exp(-\pi \dist(\vecx,\lat)^2) \,\, \forall \vecx$
\end{lemma}

\begin{proof}
  Since $\rho(\vecx+\lat) = \rho(\vecx-\lat)$:
  \[ \rho(\vecx+\lat) = 1/2 (\rho(\vecx+\lat) + \rho(\vecx-\lat)). \]
  By the definition of Gaussian function:
  \begin{align*}
    \rho(\vecx+\lat)
    &= 1/2 \sum_{\vecv \in \lat} ( \exp(-\pi \length{\vecx+\vecv}^2)+ \exp(-\pi \length{\vecx-\vecv}^2)) \\
    &= 1/2 \exp(-\pi \length{\vecx}^2)  \sum_{\vecv \in \lat} \exp(-\pi \length{\vecv}^2) (\exp(-2\pi \inner{\vecx,\vecv})+\exp(2\pi \inner{\vecx,\vecv})).
  \end{align*}
  By the Cauchy-Schwarz inequality
  $\exp(-2\pi \inner{\vecx,\vecv})+\exp(2\pi \inner{\vecx,\vecv}) \geq
  2$. We have:
  \[ \rho(\vecx+\lat) \geq \exp(-\pi \length{\vecx}^2) \sum_{\vecv \in
      \lat} \exp(-\pi \length{\vecv}^2) = \exp(-\pi \length{\vecx}^2)
    \rho(\lat).\] Therefore,
  \[ f(\vecx) = \rho(\vecx+\lat)/ \rho(\lat) \geq \exp(-\pi
    \length{\vecx}^2).\] Now for all points in the coset
  $\vecx + \lat$, there is a point $\vecx_0$ such that
  $ \length{\vecx_0}=d(\vecx_0,\lat) = d(\vecx,\lat)$. Represent
  $\vecx + \lat$ as $\vecx_0 + \lat$ and we have:
  \[ f(\vecx) = \rho(\vecx_0+\lat)/ \rho(\lat) \geq \exp(-\pi
    \length{\vecx_0}^2) = \exp(-\pi d(\vecx_0,\lat)^2) = \exp(-\pi
    d(\vecx,\lat)^2).\]
\end{proof}
From the above lemma, we see that if $d(\vecx,\lat) = \mathcal{O}(1)$,
then $f(\vecx) = \Omega(1)$, so $f$ satisfies our first requirement.

\begin{lemma}
  For any coset $\vecx + \lat$,
  $\rho((\vecx + \lat)\backslash \sqrt{n}\ball) \leq 2^{-n}
  \rho(\lat)$, where $\ball$ represents the unit ball.
\end{lemma}

\begin{proof}
  From the Claim \ref{claim2} above,
  $2^n \rho(\lat) \geq \rho_2(\vecx + \lat)$, therefore:
  \begin{align*}
    2^n \rho(\lat)
    &\geq \rho_2(\vecx + \lat) \\
    &\geq \rho_2((\vecx + \lat)\backslash \sqrt{n}\ball) \\
    &= \sum_{\vecv \in (\vecx + \lat) \backslash \sqrt{n}\ball} \exp(-\pi \length{\vecv}^2/4) \\
    &= \sum_{\vecv \in (\vecx + \lat) \backslash \sqrt{n}\ball} \exp(3\pi \length{\vecv}^2/4)\exp(-\pi \length{\vecv}^2) \\
    &\geq \exp(3\pi n/4) \sum_{\vecv \in (\vecx + \lat) \backslash \sqrt{n}\ball}\exp(-\pi \length{\vecv}^2) \\
    &\geq  4^n \rho_1((\vecx + \lat)\backslash \sqrt{n}\ball).
  \end{align*}
  Therefore,
  $\rho((\vecx + \lat)\backslash \sqrt{n}\ball) \leq 2^{-n}
  \rho(\lat)$ as desired.
\end{proof}

\begin{corollary}
  If $d(\vecx,\lat) \geq \sqrt{n}$ then $f(\vecx) \leq 2^{-n}$.
\end{corollary}

\begin{proof}
  If $d(\vecx,\lat) \geq \sqrt{n}$,
  $\vecx + \lat = (\vecx + \lat) \backslash \sqrt{n}\ball$, since
  there is no point of coset $\vecx + \lat$ in the ball radius
  $\sqrt{n}$ centered at the origin. Therefore,
  \[ \rho(\vecx + \lat) = \rho((\vecx + \lat)\backslash \sqrt{n}\ball)
    \leq 2^{-n} \rho(\lat), \] And $f(\vecx) \leq 2^{-n}$ as desired.
\end{proof}

Next we need to show that $f$ can be approximately computed.
\begin{lemma}
  $f$ can be approximated efficiently.
\end{lemma}

\begin{proof}
  The idea is we can represent $f$ by its Fourier series
  $\hat{f}(\vecw)$, with $\vecw \in \lat^*$. We know that:
  \[ \hat{f}(\vecw) = \hat{\rho}(\vecw) \det(\lat^*) / \rho(\lat) =
    \hat{\rho}(\vecw)/\rho(\lat^*).\] Therefore,
  $\sum_{\vecw \in \lat^*} \hat{f}(\vecw) = 1$, and we can view
  $\hat{f}$ as a probability distribution. Then the expectation:
  \[ \E_{\vecw \leftarrow \hat{f}}(\exp(2 \pi i \inner{\vecx,\vecw} ))
    = \sum_{\vecw \in \lat^*}\hat{f}(\vecw) \exp(2 \pi i
    \inner{\vecx,\vecw} ) = f(\vecx). \] The first equality is just
  the definition of expectation, and the second one follows from the
  Poisson Summation Formula. Therefore, to approximate $f$, we can
  estimate the expectation above. We preprocess the lattice by output
  many values of $\vecw$ from the probability distribution $\hat{f}$,
  and store them. Then, given a query point $\vecx$, we can just
  compute average value of $\exp(2 \pi i \inner{\vecx,\vecw} )$ to be
  an estimator of $f(\vecx)$.
\end{proof}

\bibliography{common/lattices,common/crypto}
\bibliographystyle{common/alphaabbrvprelim}

\end{document}


%%% Local Variables:
%%% mode: latex
%%% TeX-master: t
%%% End:
